
%目次のレイアウト調整
\setcounter{tocdepth}{4}
%-----
%----------
\renewcommand{\thefootnote}{\textasteriskcentered\arabic{footnote}} %脚注の記号変更
%==========
% パッケージ集
%==========
%数式フォント
%\usepackage{lxfonts}
\usepackage{bm}%太字のベクトルを表示
\usepackage{mathrsfs}%アルファベット筆記書体など

%欧文フォント
%\usepackage[LGR,LGRx,T2A,T1,OT1]{fontenc}
\usepackage[LGR,T2A,T1]{fontenc}
% \usepackage[utf8x]{inputenc}
\usepackage{tgheros}
\usepackage{lmodern}
%\usepackage{mathpazo}
%ギリシャ語フォント
\usepackage[polutonikogreek,greek,english,japanese]{babel}
% ¥usepackage[boldLipsian,10pt,GlyphNames]{teubner}
\usepackage{teubner}
\usepackage{substitutefont}
\substitutefont{LGR}{\rmdefault}{porson}

\usepackage[unicode,hidelinks,pdfusetitle]{hyperref}
%---------
%\languageattribute{greek}{polutoniko}
%----------
\usepackage{graphicx} %画像の挿入
\usepackage{wrapfig} %図の回り込み
\usepackage{framed,color} %枠付き文書
\definecolor{shadecolor}{gray}{0.85}
% \begin{oframed} 文 \end{oframed}
% \begin{shaded} 文 \end{shaded}
%----------
\usepackage{setspace}
%----------
\usepackage{colortbl}
\usepackage{tikz}
\usepackage{tikz-cd} %可換図式
\usepackage{tikz-3dplot}
\usepackage{pgfplots}
\pgfplotsset{compat=1.12}
%---------
\usepackage{ascmac}
\usepackage{amsmath}
\usepackage{amssymb}
\usepackage{amsthm}
\usepackage{amsfonts}
\usepackage{amscd}
% \usepackage{pb-diagram} %可換図式
\usepackage{fancybox}
\usepackage{enumerate}
\usepackage{ulem} %下線
%---------
\usepackage{mathdots} %斜めドット
\usepackage{bbold} % \mathbb{1} など
%---------
\usepackage{lipsum} %ダミーテキスト 
% 例: \lipsum[1-5]
%---------
%丸囲い
\newcommand*\circled[1]{\tikz[baseline=(char.base)]{
            \node[shape=circle,draw,inner sep=2pt] (char) {#1};}}
%---------
\renewcommand{\textgt}[1]{\textsf{\textbf{#1}}}
%---------
\usepackage{multicol} % n段組み 
%\begin{multicols}{段数} 文章 \end{multicols}
%----------
\usepackage{vwcol}
% \begin{vwcol}[widths={0.6,0.4},rule=0.5pt] 
% 文章
% \end{vwcol}
%----------
\usepackage{physics} %便利なパッケージ
% \qty() , \qty|| ,\norm{} など ...... カッコ
% \vb{} ...... 太字のベクトル
% 数式数式  \qq{テキスト}  数式数式
% \dd[指数]{x} ...... dx
% \dv{x} ............ d/dx
% \pdv{x} ........... ∂/∂x
% \dv{f}{x} ......... df/dx
% \pdv{f}{x} ........ ∂f/∂x
% \mqty(a&b\\c&d) , \mqty|| 行列・行列式
% \mqty{\imat{n}} ...... 単位行列
% \mqty{\dmat[0]{n}} ... 対角行列
% \mqty{\xmat*{a}{m}{n}} ... {a_mn}行列
%----------
\usepackage{keyval,physics2}
\usephysicsmodule{ab,ab.braket}
% \ab() , \ab<> , \ab[] カッコ
%----------
\usepackage[version=4]{mhchem} %化学反応式
\usepackage{expl3}
\usepackage{calc}
% \ce{6 CO2 + 12 H2O -> C6H12O6 + 6 O2 + 6 H2O}
%----------
% Mathematical Logic
\usepackage{turnstile}
% \renewcommand{\vdash}{\sststile{}{}}
% \renewcommand{\vDash}{\sdtstile{}{}}
% \renewcommand{\Vdash}{\dststile{}{}}
\usepackage{bussproofs}
\EnableBpAbbreviations
\renewcommand{\fCenter}{{}\Rightarrow{}}
\renewcommand{\RL}[1]{\RightLabel{{\scriptsize (#1)}}}
%%%%%%%%%% 自然演繹 
% \begin{prooftree}
% 0 \AXC { $ ... $ }  
% 1 \UIC { $ ... $ } 
% 2 \BIC { $ ... $ } 
% 3 \TIC { $ ... $ } 
% \end{prooftree}
%%%%%%%%%% シークエント計算
%\begin{prooftree}
% 0 \AX $ ...\fCenter... $
% 1 \UI $ ...\fCenter... $
% 2 \BI $ ...\fCenter... $
% 3 \TI $ ...\fCenter... $
%\end{prooftree}
%----------
\usepackage{qtree} %タブローパッケージ
%----------
\renewcommand{\labelitemi}{~$\bullet$~}
%\renewcommand{\labelitemii}{~\circ~}
\renewcommand{\labelitemii}{~$\triangleright$~}
%----------
%\usepackage{emoji} %絵文字
% \setemojifont{EmojiOneMozilla}
%\setemojifont{Noto Emoji Regular}
%----------
\newtheorem{definition}{定義}[section]
\newtheorem{proposition}[definition]{命題}
\newtheorem{theorem}[definition]{定理}
\newtheorem{lemma}[definition]{補題}
\newtheorem{corollary}[definition]{系}
\newtheorem{example}[definition]{例}
\newtheorem{practice}[definition]{演習問題}
\newtheorem*{longproof}{証明}
\newtheorem*{answer}{解答}
\newtheorem*{supplement}{補足}
\newtheorem*{remark}{注意}
%----------
% 定理環境(tcolorbox)
\usepackage{tcolorbox} %箱
\tcbuselibrary{breakable,skins,theorems}
%----------
\tcolorboxenvironment{definition}{
	blanker,breakable,
	left=3mm,right=3mm,
	top=2mm,bottom=2mm,
	before skip=15pt,after skip=20pt,
	borderline vertical={0.5pt}{0pt}{black}
}
\newtcolorbox{emptydefinition}{
	blanker,breakable,
	left=3mm,right=3mm,
	top=2mm,bottom=2mm,
	before skip=15pt,after skip=20pt,
	borderline vertical={0.5pt}{0pt}{black}
}
%----------
\tcolorboxenvironment{proposition}{
	blanker,breakable,
	left=3mm,right=3mm,
	top=3mm,bottom=3mm,
	before skip=15pt,after skip=15pt,
	borderline={0.5pt}{0pt}{black}
}
\newtcolorbox{emptyproposition}{
	blanker,breakable,
	left=3mm,right=3mm,
	top=3mm,bottom=3mm,
	before skip=15pt,after skip=15pt,
	borderline={0.5pt}{0pt}{black}
}
%----------
\tcolorboxenvironment{theorem}{
	blanker,breakable,
	left=3mm,right=3mm,
	top=3mm,bottom=3mm,
    sharp corners,boxrule=0.6pt,
	before skip=15pt,after skip=15pt,
	borderline={0.5pt}{0pt}{black},
    borderline={0.5pt}{1.5pt}{black}
}
\newtcolorbox{emptytheorem}{
	blanker,breakable,
	left=3mm,right=3mm,
	top=3mm,bottom=3mm,
    sharp corners,boxrule=0.6pt,
	before skip=15pt,after skip=15pt,
	borderline={0.5pt}{0pt}{black},
    borderline={0.5pt}{1.5pt}{black}
}
%----------
\tcolorboxenvironment{lemma}{
	blanker,breakable,
	left=3mm,right=3mm,
	top=3mm,bottom=3mm,
	before skip=15pt,after skip=15pt,
	borderline={0.5pt}{0pt}{black}
}
%----------
\tcolorboxenvironment{corollary}{
	blanker,breakable,
	left=3mm,right=3mm,
	top=3mm,bottom=3mm,
	before skip=15pt,after skip=15pt,
	borderline={1.0pt}{0pt}{black,dotted}
}
\newtcolorbox{emptycorollary}{
	blanker,breakable,
	left=3mm,right=3mm,
	top=3mm,bottom=3mm,
	before skip=15pt,after skip=15pt,
	borderline={1.0pt}{0pt}{black,dotted}
}
%----------
\tcolorboxenvironment{example}{
	blanker,breakable,
	left=3mm,right=3mm,
	top=3mm,bottom=3mm,
	before skip=15pt,after skip=15pt,
	borderline={0.5pt}{0pt}{black}
}
%----------
\tcolorboxenvironment{practice}{
	blanker,breakable,
	left=3mm,right=3mm,
	top=3mm,bottom=3mm,
	before skip=15pt,after skip=15pt,
	borderline={0.5pt}{0pt}{black}
}
%----------
\tcolorboxenvironment{proof}{
	blanker,breakable,
	left=3mm,right=3mm,
	top=2mm,bottom=2mm,
	before skip=15pt,after skip=20pt,
	% borderline west={1.5pt}{0pt}{black,dotted}
	borderline vertical={1pt}{0pt}{black,dotted}
	% borderline vertical={0.8pt}{0pt}{black,dotted,arrows={Square[scale=0.5]-Square[scale=0.5]}}
	}
%----------
\tcolorboxenvironment{supplement}{
	blanker,breakable,
	left=3mm,right=3mm,
	top=2mm,bottom=2mm,
	before skip=15pt,after skip=20pt,
	% borderline west={1.5pt}{0pt}{black,dotted}
	% borderline vertical={0.5pt}{0pt}{black,arrows = {Circle[scale=0.7]-Circle[scale=0.7]}}
	borderline vertical={0.5pt}{0pt}{black}
	% borderline vertical={0.5pt}{0pt}{black},
	% borderline north={0.5pt}{0pt}{white,arrows={Circle[black,scale=0.7]-Circle[black,scale=0.7]}}
	}
%----------
\tcolorboxenvironment{remark}{
	blanker,breakable,
	left=3mm,right=3mm,
	top=1mm,bottom=1mm,
	before skip=15pt,after skip=20pt,
	% borderline west={1.5pt}{0pt}{black,dotted}
	% borderline vertical={0.5pt}{0pt}{black,arrows = {Circle[scale=0.7]-Circle[scale=0.7]}}
	borderline vertical={0.5pt}{0pt}{black}
	% borderline vertical={0.5pt}{0pt}{black},
	% borderline north={0.5pt}{0pt}{white,arrows={Circle[black,scale=0.7]-Circle[black,scale=0.7]}}
	}
%----------
% マークシート記号
\newcommand{\egg}[1]{\raisebox{-3pt}{
	\begin{tikzpicture}[x=1pt,y=1pt,line width=1pt]
		\draw (0,0) ellipse (4.5 and 6);
		\draw (0,0) node {
			\usefont{T1}{phv}{m}{n}
			\fontsize{9pt}{0}\selectfont #1 \/};
	\end{tikzpicture}}}
% マークシート記号
\newcommand{\eggg}[1]{\raisebox{-3pt}{
	\begin{tikzpicture}[x=1pt,y=1pt,line width=1pt]
		\draw[fill=black!30] (0,0) ellipse (4.5 and 6);
		\draw (0,0) node {
			\usefont{T1}{phv}{m}{n}
			\fontsize{9pt}{0}\selectfont #1 \/};
	\end{tikzpicture}}}
%-------------------- 
