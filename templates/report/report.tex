\documentclass[uplatex,a4paper,11pt,dvipdfmx]{jsarticle}
\usepackage[dvipdfmx]{graphicx}
\usepackage{amsmath}
\usepackage{ascmac}
\usepackage{float}
\usepackage{listings,jvlisting}
\lstset{
  language=Python,
  basicstyle={\ttfamily},
  identifierstyle={\small},
  commentstyle={\smallitshape},
  keywordstyle={\small\bfseries},
  ndkeywordstyle={\small},
  stringstyle={\small\ttfamily},
  frame={tb},
  breaklines=true,
  columns=[l]{fullflexible},
  numbers=left,
  xrightmargin=0zw,
  xleftmargin=3zw,
  numberstyle={\scriptsize},
  stepnumber=1,
  numbersep=1zw,
  lineskip=-0.5ex,
  tabsize=4
}
\renewcommand{\lstlistingname}{プログラム}
\begin{document}
\begin{enumerate}
    \item {\bf \large 実験目的}\\した角度成する. \\
    
    \item {\bf \large 原理}\\ミング左手の法則による. \\

    \item {\bf \large 【実験1】モータードライバの入出力端子の確認}\\
    \begin{enumerate}
        \item[3.1] 実験方法\\
        実験の前に以下のプ Piに転送する. \\
        
        \begin{itembox}{転送コマンド}
        {\bf scp *.py student@raspberrypi.local:\(\sim\)/work/2\_motors}
        \end{itembox}
        
        回路図を以下に示す.jjjjjjjjjjjjjjjjjj
        \\

        \item[3.2] 結果\\
        モータードライバーの変化のようすを示す.

        \item[3.3] 検討\\
        3.3.1 IN1とIN2のスイッチングはどのように行っているか\\

        3.3.2 \\

        3.3.3\\

    \end{enumerate}
    
    \item {\bf \large 【実験2】Raspberry Piとモータードライバを用いた制御}\\
    \begin{enumerate}
        \item[4.1]実験方法\\
        以下の図のような回路を作る. GPIして実行する. \\

        \item[4.2] 結果\\
        今回の実験で作ったプログラムを示す.\\

        \item[4.3] 検討\\
        
        4.3.1\\

        4.3.2\\
        
    \end{enumerate}
    \item {\bf \large 【実験3】Raspberry Piを用いた制御}\\
    \begin{enumerate}
    \item[5.1]実験方法\\
    5のときのデューティ比が先程の計算と合っているか確かめる. 

    \item[5.2] 結果\\
    今回の実験で作成したプログラムを以下に示す. 
    \begin{lstlisting}[caption=gpioSwitching.py, label=3]
    
    \end{lstlisting}
    

    \item[5.3] 検討\\
        プログラムのフローチャートを以下
        
    \end{enumerate}
    {\bf \large 以下からテーマ10}
    \item {\bf \large 【実験4】サーボモーターの回転角度と制御信号の関係}\\
    \begin{enumerate}
        \item[6.1]実験方法\\
        以下の図のように回ーティ比を求める. \\

        \item[6.2] 結果\\
        fff

        \item[6.3] 検討\\
        フローチャートを以下に示す
        \begin{figure}[H]
            \centering
            \includegraphics[width=9cm]{jikken4.jpg}
            
        \end{figure}
    \end{enumerate}
    \item {\bf \large 【実験5】半固定抵抗によって回転角度を操作する制御}\\
    \begin{enumerate}
        \item[7.1]実験方法\\
        回路図はを確かめる. \\

        \item[7.2] 結果\\
        angleControl.pyを変更した部分を示す.
        
        angleControl.pyによって可変抵
        
    \end{enumerate}
    \item {\bf \large 【実験6】明るさによって回転角度を操作する方法}\\
    \begin{enumerate}
        \item[8.1]実験方法\\
        以下の図のように回とを確認する. 

        \item[8.2] 結果\\
        
    \end{enumerate}
    \item {\bf \large 結論}\\
    \quad rr
    
\end{enumerate}

\begin{thebibliography}{9}
    \bibitem{suken} [104 数研 物理 313]改訂版 物理, 数研出版, (2023.7,17)
    \bibitem{rika} 理科大事典
    \bibitem{hbri} Hブリッジ回路によるブラシ付DCモーターの駆動:原理, Tech Web, (2018,8.28)
    \bibitem{logic} ロジックICとは? , 東芝デバイス\&ストレージ会社, (2023,7.18)
\end{thebibliography}
\end{document}
