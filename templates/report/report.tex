\documentclass[uplatex,a4paper,11pt,dvipdfmx]{jsarticle}
\usepackage[dvipdfmx]{graphicx}
\usepackage{amsmath}
\usepackage{ascmac}
\usepackage{float}
\usepackage{listings,jvlisting}
\lstset{
  language=Python,
  basicstyle={\ttfamily},
  identifierstyle={\small},
  commentstyle={\smallitshape},
  keywordstyle={\small\bfseries},
  ndkeywordstyle={\small},
  stringstyle={\small\ttfamily},
  frame={tb},
  breaklines=true,
  columns=[l]{fullflexible},
  numbers=left,
  xrightmargin=0zw,
  xleftmargin=3zw,
  numberstyle={\scriptsize},
  stepnumber=1,
  numbersep=1zw,
  lineskip=-0.5ex,
  tabsize=4
}
\renewcommand{\lstlistingname}{プログラム}
\begin{document}
\begin{enumerate}
    \item {\bf \large 実験目的}\\した角度成する. \\
    
    \item {\bf \large 原理}\\ミンよる. \\

    \item {\bf \large 確認}\\
    \begin{enumerate}
        \item[3.1] 実験方法\\
        実験の前に以る. \\
        
        \begin{itembox}{転送コマンド}
        {\bf scp *.py sturs}
        \end{itembox}
        
        回路図を以下に示す.jjjjjjjjjjjjjjjjjj
        \\

        \item[3.2] 結果\\
        モ示す.

        \item[3.3] 検討\\
        3.3.1 INか\\

        3.3.2 \\

        3.3.3\\

    \end{enumerate}
    
    \item {\bf \large 【御}\\
    \begin{enumerate}
        \item[4.1]実験方法\\
        以下の図る. \\

        \item[4.2] 結果\\
        今回の示す.\\

        \item[4.3] 検討\\
        
        4.3.1\\

        4.3.2\\
        
    \end{enumerate}
    \item {\bf \large 御}\\
    \begin{enumerate}
    \item[5.1]実験方法\\
    5のときいるか確かめる. 

    \item[5.2] 結果\\
    今回の実験で作に示す. 
    \begin{lstlisting}[caption=gpioSwitching.py, label=3]
    a
    \end{lstlisting}
    

    \item[5.3] 検討\\
        プログ
        
    \end{enumerate}

\end{enumerate}

\begin{thebibliography}{9}
    \bibitem{suken} 著者名,書名, 出版社, (出版年)
    \bibitem{rika} 理科大事典
\end{thebibliography}
\end{document}