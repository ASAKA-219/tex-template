\documentclass[a4paper, amsfonts, amssymb, amsmath, reprint, showkeys, 
twocolumn]{revtex4-1}

\usepackage{amsthm}
\usepackage{mathtools}
\usepackage{physics}
\usepackage{xcolor}
\usepackage[dvipdfmx]{graphicx}
\usepackage[left=13mm,right=13mm,top=25mm,columnsep=15pt]{geometry} 
\usepackage{adjustbox}
\usepackage{placeins}
\usepackage[T1]{fontenc}
\usepackage{lipsum}
\usepackage{csquotes}

\renewcommand{\figurename}{Fig.}%図をFigに
\renewcommand{\tablename}{Table.}%表をTableに

\bibliographystyle{apsrev4-1}
\begin{document}
\title{論文フォーマット}

\author{Yusuke Asaka}
\thanks{身分をかく}
\affiliation{Department of Information and Communication Technology, 
Abcdef University, Tokyo, 000-0000, America}
    %学科?専攻?とか所属とか

%\date{\today} % Leave empty to omit a date

\begin{abstract}
研究の概要を7〜8行以内で書く。形式によっては英語で書く
\end{abstract}

\keywords{研究のキーワードを書く}

\maketitle

1.\textbf{はじめに}\\
\quad 目的、研究に至る背景、課題を書く ,.かんまとドットで書くこと\\

2.\textbf{本論1}\\
\quad 研究の最低限の基礎知識、理論を書く\\

3.\textbf{本論2}\\
\quad 実験、シミュレーションなどを書く。\\

4.\textbf{結論}\\
\quad 実験、シミュレーションによって得られた結論を書く。\\

\textbf{謝辞}\\
\quad 本研究を実施するにあたり、様々な助言を与えてくださった方々に感謝申し上げます。\\

\begin{thebibliography}{9}
\bibitem{Griffiths}
著者, 書名(出版社, 年).

\bibitem{Friffiths}
著者, 書名(出版社, 年).
\bibitem{Hriffiths}
著者, 書名(出版社, 年).
\bibitem{iffiths}
著者, 書名(出版社, 年).
\bibitem{Wiffiths}
著者, 書名(出版社, 年).
\bibitem{pfiths}
著者, 書名(出版社, 年).

\end{thebibliography}
\end{document}