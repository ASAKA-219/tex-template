\documentclass[a4paper, amsfonts, amssymb, amsmath, reprint, showkeys, 
twocolumn]{revtex4-1}

\usepackage{amsthm}
\usepackage{mathtools}
\usepackage{physics}
\usepackage{xcolor}
\usepackage[dvipdfmx]{graphicx}
\usepackage[left=13mm,right=13mm,top=25mm,columnsep=15pt]{geometry} 
\usepackage{adjustbox}
\usepackage{placeins}
\usepackage[T1]{fontenc}
\usepackage{lipsum}
\usepackage{csquotes}

\renewcommand{\figurename}{Fig.}%図をFigに
\renewcommand{\tablename}{Table.}%表をTableに

%\bibliographystyle{apsrev4-1}
\begin{document}
\title{北条時頼の生涯と得宗専制政治の確立}

\author{Yusuke Asaka}
\thanks{あああああ}
\affiliation{Department of Information and Communication Technology, 
abcdef University, Tokyo, Japan}
    %学科?専攻?とか所属とか

%\date{\today} % Leave empty to omit a date

\begin{abstract}
2024年の7月から9月にかけて放送されて
\end{abstract}

\keywords{仏教者、評定衆、引付衆、得宗専制政治}

\maketitle

1.\textbf{はじめに}\\
\quad 北条時頼とはどんな人物だったのであろうか。その姿は大きく二つに分類される。一つは執
権としての武士の姿である。有名な木像として建長寺所蔵の北条時頼坐像を
図\ref{fig:enter-label}に示す。
    \begin{figure}[htbp]
        \centering
        \includegraphics[width=5cm]{}
        \caption{北条時頼坐像}
        \label{fig:enter-label}
    \end{figure}

とし込むことで広く時頼の功績を世に示すことを目的とする。その手始めとして、最初に執権政
治の概観を述べてから、時頼の生涯についての大まかなまとめと、得宗専制体制がどのように確
立されていったのかのまとめを述べる。そして、時頼という人物の特筆すべき点は、仏教にも関心
を示し幅広い考えを取り入れていったことにある。最後に時頼の仏教への取り組みにも触れてみたい。\\

2.\textbf{執権政治の概観}\\
\quad まずは鎌倉幕府の特徴である\textbf{執権制度}について見ていこう。
執権とは本来の用語としての意味は「政権を握る」\cite{Fleming}の意、またはその人を指す。
それが鎌倉幕府では将軍の政務を補佐し、政務を統括する役職として定着した。初めは政所の別当
を院政時代の院庁の別当にちなんで執権と呼んだが、源頼朝の義兄弟である\textbf{北条義時}が
侍所別当であった和田義盛を討滅(和田合戦)した際に侍所別当も兼ねることになり、その両職を
兼ねるものが執権であるとみなされるようになった。鎌倉幕府の発給する正式な文書の形式である
「関東下知状」の最終部分に執権の花押(署名)がされており、公式に幕府政治の重責を担って
いることが分かる。\\
\quad 執権制度の変遷は佐藤進一氏によれば、三世代に別れると言われており、
この説が有力とされている。その三世代とは、
\begin{enumerate}
    \item 将軍独裁(初代将軍源頼朝$\sim$北条時政)
    \item 執権政治(執権北条義時〜泰時)
    \item 得宗専制(時頼〜貞時)
\end{enumerate}
である\cite{Griffiths}。また、安田元久氏によればもう一世代増えて、御内人の独裁時代
(貞時、高時以降)があるという\cite{Fleming}。この区分の第二段階として執権政治は複数の
有力御家人による合議制によって始まった。初代執権の時政の頃に源頼朝が死去して以降十三人の
御家人による合議制が始まり、その子義時は有力御家人を次々と滅ぼして執権独裁体制を築くが、
三代目泰時の代になって評定衆を作っており、合議体制の中心としての役割を果たしていたことが分
かる。しかし、泰時の孫の時頼の時代になって御家人の合議体製は終わりを迎え、代わりに北条氏
一門の当主である得宗を中心に北条氏が政治を独裁する\textbf{得宗専制政治}が始まっていく。
これが第三段階である。\\
\quad 日本中世史研究家たちの見解として、「狭義の執権政治は、その基本的性質の一つである
合議体制が、その実質を失う時点において崩壊する」と述べている\cite{Griffiths}。つまり、
得宗専制になった時点で執権政治は変質し、違う政治体制であることを意味している。こうなった
原因には、執権政治そのものの持つ矛盾があった。それを解決するために独裁的政治をせざるを得な
かったと考えられる。\\
\quad 得宗専制政治になったことで得宗権力は安定したものの、中央集権的すぎて御家人の不
満が高まり、後の北条高時の時代になって鎌倉幕府は滅亡する。具体的な要因は元寇の戦後に戦っ
た武士たちに十分に恩賞を取らすことができず、所領を分割相続していたこともあって御家人の生
活が困窮し、新たな武家政権を望む声が高まっていったことによる。\\
\quad ここまで執権政治の概要を見てきたが、時頼の行った得宗専制体制の確立によって執権政
治が大きく変質したことが分かるであろう。次の章では時頼の半生を概観し、執権政治の変遷に与
えた影響を見ていく。\\

3.\textbf{執権専制の確立ー時頼の苦難ー}\\
\quad 北条時頼は一生を鎌倉で過ごしたと思われがちだが、実は京都の六波羅で生まれている。
時頼は1227年(嘉禄三年)、北条時氏の次男として生まれた。父親の時氏は当時の3代目執権の泰
時から目をかけられ、六波羅探題北方に任じられたり、従四位下に推挙されたりと順調に次期執権と
して育てられたが、病気により1230(寛喜二年)年に死んでしまった。時頼はその三歳上の兄経時と
共に安達景盛の女(松下禅尼)に育てられた。泰時も時氏の死後には兄弟の育成に心血を注ぎ、様々
な教訓を与えている。その逸話として名高いのが、三浦氏と小山氏の争いが起きたとき、経時は三浦
氏に理があると判断し三浦泰村に手勢を派遣したが、時頼は事態を静観するだけで何もしなかった。
この事態に泰時は、「時頼の態度は誠に神妙である」として時頼に褒美を授けた。この逸話は時頼
が経時の跡を継いだ正当性を与えるために脚色された記事であると考えられるが、時頼の兄へ遠慮し
て表立った行動を避けている描写の一つである。\\
\quad その後泰時が1242年(仁治三年)に死去すると兄の経時が四代目執権になった。経時の執権
就任時には将軍九条頼経が求心力を発揮しており、反執権派が頼経を担いで経時を追い落とそうとす
る危険があった。そのため、経時は将軍九条頼経を京都に送還し、代わりに子の九条頼嗣を将軍に迎
えた。それからすぐ経時は病気になり、執権を時頼に譲った。そして1246年(寛元四年)に二十三歳
で死去した。いよいよ執権時頼の時代の始まりである。\\
\quad しかし、将軍が頼嗣に代わっても依然として前将軍頼経は大殿として権勢があった。そこに
以前から不満を持っていた名越光時がクーデターを起こす(\textbf{寛元の政変})。時頼はすぐ
謀反を察知し、名越光時が出家することで合戦とはならなかった。事件の後これに関わったとされる
関東申次の頼経の父、九条道家は更迭された。\\
\quad この事件は終わったがまだまだ火種は燻っていた。寛元の政変でも将軍側にいた三浦光村が
事件には関わっていなかったため、処罰されずに有力御家人の地位を占めていた。これに時頼は
警戒した。翌年1247年に改元され宝治となった。その頃三浦氏と安達氏の有力御家人の座を巡る
争いが表面化し出した。安達氏は先に述べたように時頼の外戚の地位にいるため、三浦氏が邪魔で
あった。そのため安達景盛が義景に発破をかけ、三浦氏の暴走を誘発する行動に出た。初め時頼は
三浦氏を信じていたようであるが、外戚である安達氏が積極策に出たため、時頼も三浦氏討伐に心を
変え、三浦氏の邸宅に兵を向けた。これによって三浦泰村以下一族は自害し、滅んだ
(\textbf{宝治合戦})。この事件によって有力な御家人はいなくなり、北条得宗家で幕府の評定
衆は占められる結果となった。\\
\quad 寛元、宝治の二つの政変によって北条氏の専制体制がさらに強まり、執権政治は成熟を迎
えるのである。政変で北条氏に並ぶ御家人である三浦氏、得宗家の地位を脅かす名越氏、有力官僚
であった後藤、三善氏らの一族が次々と追い落とされたことで、執権に政治的な権力がさらに集中
している。\\

4.\textbf{執権時頼の政策について}\\
\quad 一応時頼の行った改革について簡単に見ていこう。まずは御家人保護政策である。代表的な
のは\textbf{引付衆}を組織したことであろう。引付とは、評定衆の下に置かれた新たな裁判機関
である。佐藤進一氏によれば、引付の設置は裁判の迅速化が最大の理由であり、くわえて、人心を安
心させ新執権時頼への心服を得る方策の一つとして訴訟制度の改革が行われた、という(『鎌倉幕府
訴訟制度の研究』)。引付は三番制であり、それぞれの番(訴訟の審理担当チーム)ごとに頭人が置
かれ、その下に数名の評定衆、引付衆、奉行人が配置されていた。また、引付の職務は、裁判の中で
も御家人相互間の訴訟や、荘園領主と御家人間の訴訟を審理することであった。裁判の迅速な審理が
求められ、時頼の信頼が厚くなった。\\
\quad 時頼は、引付設置の他にも様々な訴訟制度の改革を進めている。訴人と論人で三回答弁を行
い、それで決着が付かなければ問注所で当人同士が話し合うという三問三答の形式を作った。また、
引付の文書審査で、訴訟の善悪が明らかである場合にはその審理を省略することが定められた。
また、評定衆の勤務態度も怠けていると罰が加えられた。御家人の精勤の奨励も、時頼の政策の
一つである。\\
\quad 他にも御家人保護政策は多数行っており、
\begin{enumerate}
    \item 京都大番役の御家人が篝屋を兼ねていたため、大番役の当番表を作り、3ヶ月で交代
    させた。
    \item 問注所で裁判する役人にも当番表を作り、遅刻しない、裁判は巳の刻(午前10時)
    までに始めることを指示
\end{enumerate}
このように御家人の立場を保障するような決まりを多く作っていたが、その代わりに御家人には誠心
誠意取り組むことを求めていた。\\
\quad また、時頼は御家人の働きぶりの監視などの意味もあって撫民政策も多く展開した。
その一つとして、庶民の地頭への訴訟は自由に行えたが、庶民同士の裁判は地頭に処理させている。\\

5.\textbf{時頼の仏教との関わり}\\
\quad 時頼は仏教に厚く帰依していた。時頼の僧形の姿の木像を下図に示す。
\begin{figure}[h]
    \centering
    \includegraphics[width=4cm]{}
    \caption{最明寺所蔵 北条時頼木像}
\end{figure}信仰していた宗派として
\begin{itemize}
    \item 禅宗
    \item 律宗
    \item 浄土宗
    \item 真言密教
    \item 日蓮宗
\end{itemize}
などがある。時頼政権の安定してきた宝治合戦(1247年)後、密教修法をする護持僧として
\textbf{隆弁}という僧に絶大な信頼を寄せ、鶴岡八幡宮の別当に任命するなどその後も重用
した。また、1248年頃に来日してから気にかけていた禅宗の\textbf{蘭渓道隆}への入れ込みが
激しく、何度も連絡を取っていた。そのため、出家以前から蘭渓道隆のところを訪ね、教えを聞い
ていた。蘭渓道隆のために建長寺を建立し、自らその檀那になり、その開山として蘭渓道隆を置い
た。さらに、弘長2年頃からは\textbf{叡尊}にも入れ込んでおり、何度も授戒させて欲しいと
手紙を出している。その一番弟子、忍性がその跡を継いでいる。\\
\quad 他にも円爾、日蓮とも交流があり、前者とは蘭渓道隆と絡んで禅宗のことについて手紙を
やりとりしている。日蓮の方は、彼の念仏の規制などの排他的な提言は聞き入れなかったものの、
彼の教えに関してはある程度関心を持っており、日蓮は「時頼は話のわかる人だ」という旨の書状を
後に残している。\\

6.\textbf{時頼の最期}\\
\quad 1256年(康元元年)、宗尊親王や北条重時の出家、九条頼経、頼嗣の死去によってついに
時頼は出家し、執権を北条長時に譲った。しかし、執権職を辞した後も正月のおうばん(非常用漢
字)沙汰人を重時と共に務めており、幕府内の実力者の地位は執権、連署に依存しないことが示さ
れた。これによって時頼は出家しても執権の地位とは無関係に幕政の上に君臨する存在であったと
いえ、「得宗専制」と言える状況であった。さらに、1261年(弘長元年)、長らく幕府の重鎮であ
った北条重時が死去した。それから二年後の1263年、時頼は病に伏し、体調が戻らず亡くなった。
三十七歳であった。\\

7.\textbf{時頼の人格}\\
\quad ここで、北条時頼という人物を描く上で重要な時頼の性格、行動についてまとめておく。
時頼は永井路子氏によれば、「二面性がある」、「小説で書くと良い人だか悪い人だか分からない
と言われる」と評されている。しかし、高橋慎一朗氏によれば、「真面目で責任感の強い人物」と
言われる\cite{Griffiths}。実際、祖父の北条泰時が存命中に様々な儀式をそつなくこなしてい
たり、三浦氏と小山氏の争い中の行動も静観していたりと、冷静で慎重な性格という評価が的を得
ていると本研究では解釈する立場を取る。\\

8.\textbf{まとめと結論}\\
\quad これまで時頼の生涯と執権としての取り組み、仏教への取り組みを見てきた。その結論
としては時頼はより良い政治を目指して執権への権力集中を確実にし、執権を中心とする御家人支
配構想をある程度完成させた人物と言える。しかし、執権に権力を集中させたことによって執権制
度自体の意義は完全に無くなり、北条氏の権力独占に傾いていくことになった。これが結果的に元寇
の戦後に所領を増やせなかった武士の反感を買い、幕府の滅亡に繋がっていくのだが、それは時頼に
は関係ない事柄である。また、武士の棟梁として御家人の保護政策を行ったり、質素倹約を呼びかけ
ているところが着実に政務をこなしてきた生真面目な堅物という印象も受けた。\\
\quad また、まとめとして時頼の人生を振り返ると、執権の権力確立に努力した人生であった。真
面目で慎重な性格で、仕事に忠実な人という印象がある。なので、高橋慎一朗氏は「仕事に関して
は信頼できるが、個人的におつきあいするのはちょっとご遠慮したい、真面目な優等生タイプ」だ
と述べている\cite{Griffiths}。また、時頼は気配りの人という評価もある。宝治合戦の時の
三浦氏への煮え切らない態度もそうであるし、寛元の政変での対応もそうであるように、なるべく
争いを回避しようと気配りをしている様が散見される。そういう性格であったから、得宗専制政治に
舵を切る時も煮え切らなくてストレスが多かったものと推測される\cite{Griffiths}。その結果
として仏教に深く帰依して心の救いを求めていても不思議ではない。このように、ストレスの多い
苦悩の人生だったのではなかろうか。\\

\textbf{謝辞}\\
\quad 本研究を実施するにあたり、様々な助言を与えてくださった方々に感謝申し上げます。

\begin{thebibliography}{9}
\bibitem{Griffiths}
高橋慎一朗, 人物叢書 北条時頼(吉川弘文館, 2013).

\bibitem{Fleming}
安田元久,鎌倉執権政治ーその展開と構造(教育社,1979).

\end{thebibliography}
\end{document}
